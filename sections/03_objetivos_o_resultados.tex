% Para poder usar footnotes en el título de la sección:
% \SECTION[Título interno de la sección]{Título visible de la sección}
\section[Objetivos Generales o Resultados de Aprendizaje]{Objetivos Generales o Resultados de Aprendizaje\footnote{Texto íntegro de los objetivos o resultados de aprendizaje, tomado del programa de la asignatura.}}
%Eliminar \PENDIENTE en producción.
\pendiente{Según el enfoque de la asignatura, se pueden definir objetivos generales o resultados de aprendizaje. Pero no los dos a la vez. Revise el programa de la asignatura.}

\small
\begin{xltabular}{\linewidth}{@{}X X X@{}}
    \toprule
    \encabezadodetabla{Conceptuales} & \encabezadodetabla{Procedimentales} & \encabezadodetabla{Actitudinales} \\
    \midrule
    \begin{itemize}[labelsep=2pt, itemsep=0.6em]
        \item \lipsum[1][1-3] % Texto de ejemplo, eliminar en producción
        \item \lipsum[2][1-3] % Texto de ejemplo, eliminar en producción
        \item \lipsum[3][1-3] % Texto de ejemplo, eliminar en producción
        \item \lipsum[4][1-3] % Texto de ejemplo, eliminar en producción
    \end{itemize} &
    \begin{itemize}[labelsep=2pt, itemsep=0.6em]
        \item \lipsum[1][1-3] % Texto de ejemplo, eliminar en producción
        \item \lipsum[2][1-3] % Texto de ejemplo, eliminar en producción
        \item \lipsum[3][1-3] % Texto de ejemplo, eliminar en producción
        \item \lipsum[4][1-3] % Texto de ejemplo, eliminar en producción
    \end{itemize} &
    \begin{itemize}[labelsep=2pt, itemsep=0.6em]
        \item \lipsum[1][1-3] % Texto de ejemplo, eliminar en producción
        \item \lipsum[2][1-3] % Texto de ejemplo, eliminar en producción
        \item \lipsum[3][1-3] % Texto de ejemplo, eliminar en producción
        \item \lipsum[4][1-3] % Texto de ejemplo, eliminar en producción
    \end{itemize} \\
    \bottomrule
\end{xltabular}
\pagebreak