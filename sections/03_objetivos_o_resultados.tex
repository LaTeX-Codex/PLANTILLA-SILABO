% Para poder usar \footnotes en el título de la sección:
% \SECTION[Título interno de la sección]{Título visible de la sección}
\section[Competencias a Desarrollar u Objetivos Generales]{Competencias a Desarrollar u Objetivos Generales\footnote{Texto íntegro de los objetivos o competencias a desarrollar, tomado del programa de la asignatura.}}

%! Eliminar \PENDIENTE en producción.
\pendiente{Según el enfoque de la asignatura, se pueden definir objetivos generales o compentencias a desarrollar. Pero no los dos a la vez. Revise el programa de la asignatura.}

%! Eliminar \PENDIENTE en producción.
\pendiente{La siguiente enumeración es en caso de que sea una asignatura por competencias. Si es por objetivos, elimine esta tabla y use la enumeración.}

\begin{itemize}[leftmargin=*, itemsep=0.6em]
    \item \lipsum[1][1-3] %! Texto de ejemplo, eliminar en producción
    \item \lipsum[2][1-3] %! Texto de ejemplo, eliminar en producción
    \item \lipsum[3][1-3] %! Texto de ejemplo, eliminar en producción
    \item \lipsum[4][1-3] %! Texto de ejemplo, eliminar en producción
    \item \lipsum[5][1-3] %! Texto de ejemplo, eliminar en producción
    \item \lipsum[6][1-3] %! Texto de ejemplo, eliminar en producción
    \item \lipsum[7][1-3] %! Texto de ejemplo, eliminar en producción
    \item \lipsum[8][1-3] %! Texto de ejemplo, eliminar en producción
    \item \lipsum[9][1-3] %! Texto de ejemplo, eliminar en producción
    \item \lipsum[10][1-3] %! Texto de ejemplo, eliminar en producción
\end{itemize}

%! Eliminar \PENDIENTE en producción.
\pendiente{La siguiente tabla es en caso de que sea una asignatura por objetivos. Si es por competencias, elimine esta tabla y use la primera.}

\small
\begin{xltabular}{\linewidth}{@{}X X X@{}}
    \toprule
    \thead{\small{\textbf{Conceptuales}}} & \thead{\small{\textbf{Procedimentales}}} & \thead{\small{\textbf{Actitudinales}}} \\
    \midrule
    \begin{itemize}[labelsep=2pt, itemsep=0.6em]
        \item \lipsum[1][1-3] %! Texto de ejemplo, eliminar en producción
        \item \lipsum[2][1-3] %! Texto de ejemplo, eliminar en producción
        \item \lipsum[3][1-3] %! Texto de ejemplo, eliminar en producción
        \item \lipsum[4][1-3] %! Texto de ejemplo, eliminar en producción
    \end{itemize} &
    \begin{itemize}[labelsep=2pt, itemsep=0.6em]
        \item \lipsum[1][1-3] %! Texto de ejemplo, eliminar en producción
        \item \lipsum[2][1-3] %! Texto de ejemplo, eliminar en producción
        \item \lipsum[3][1-3] %! Texto de ejemplo, eliminar en producción
        \item \lipsum[4][1-3] %! Texto de ejemplo, eliminar en producción
    \end{itemize} &
    \begin{itemize}[labelsep=2pt, itemsep=0.6em]
        \item \lipsum[1][1-3] %! Texto de ejemplo, eliminar en producción
        \item \lipsum[2][1-3] %! Texto de ejemplo, eliminar en producción
        \item \lipsum[3][1-3] %! Texto de ejemplo, eliminar en producción
        \item \lipsum[4][1-3] %! Texto de ejemplo, eliminar en producción
    \end{itemize} \\
    \bottomrule
\end{xltabular}
\pagebreak