\section{Metodología de Evaluación}

La aprobación satisfactoria del curso depende de una nota mayor o igual a 60 puntos que resulta del promedio de dos evaluaciones parciales a lo largo del curso.

La primera prueba parcial está estructurada en 60\% de evaluación sumativa y 40\% de evaluación formativa. La evaluación formativa estará dividida en dos partes: un examen teórico a realizarse en el aula virtual; y la defensa de una práctica de laboratorio basada en el contenido práctico visto en las prácticas de laboratorio del parcial.

La segunda prueba parcial también se estructura del 60\% de evaluación sumativa y 40\% de evaluación formativa enfocado en un proyecto final donde aplique todo lo aprendido.

Las evaluaciones sumativas se componen de prácticas de laboratorio y actividades de lectoescritura que resultan en exposiciones o informes de investigación de cada tema propuesto.

Las evaluaciones formativas se enfocarán en guiar al estudiante a trabajar en un proyecto de magnitud o dificultad considerable, aplicando de forma integral los conocimientos adquiridos en el transcurso de la asignatura a un caso simulado, dando una solución satisfactoria al reto presentado.

Es recomendable que las evaluaciones prácticas se valore el trabajo individual, para garantizar el aprendizaje total en cada uno.
\pagebreak