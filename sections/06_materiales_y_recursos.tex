\section{Materiales y Recursos Didácticos}

\subsection*{Materiales físicos y espacios}

\begin{itemize}
    \item \textbf{Pizarra acrílica}: Superficie para escritura y explicaciones en tiempo real.
    \item \textbf{Pantalla o proyector}: Dispositivo de visualización de contenido digital/audiovisual.
    \item \textbf{Laboratorio de computación}: Espacio equipado con equipos tecnológicos para prácticas.
    \item \textbf{Computadoras}: Dispositivos principales para ejecutar software, acceder a recursos digitales, realización de prácticas y elaboración de informes.
\end{itemize}

\subsection*{Documentación}

\begin{itemize}
    \item \textbf{Sílabo}: Documento oficial con estructura, objetivos y cronograma del curso.
    \item \textbf{Planes de clase}: Organización detallada de actividades por sesión.
    \item \textbf{Presentaciones con diapositivas}: Recursos visuales para explicar conceptos clave.
    \item \textbf{Guías de laboratorio}: Manuales con protocolos para experimentos o prácticas técnicas.
    \item \textbf{Ejercicios prácticos guiados}: Actividades estructuradas con instrucciones paso a paso.
    \item \textbf{Bibliografía de apoyo}: Libros/textos complementarios para profundizar en los temas.
\end{itemize}

\subsection*{Herramientas digitales y software}

\begin{itemize}
    \item \textbf{Plataforma virtual MOODLE}: Entorno digital para gestión de cursos y materiales educativos.
    \item \textbf{Sistema Operativo (Windows/Linux)}: Plataforma base para operar computadoras.
    \item \textbf{Microsoft Word, Google Docs, LibreOffice Writer o LaTeX}: Herramienta para creación/edición de documentos textuales.
    \item \textbf{Microsoft PowerPoint, Google Slides o LibreOffice Impress}: Software para diseñar presentaciones multimedia.
    \item \textbf{Telegram}: Aplicación de mensajería para comunicación informal con estudiantes.
    \item \textbf{Microsoft Teams}: Plataforma para videoclases, reuniones y colaboración institucional.
\end{itemize}
\pagebreak