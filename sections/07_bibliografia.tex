\section{Bibliografía}
% Aquí debe poner la bibliografía de la asignatura
%* Use el formato APA para las referencias

%! Eliminar este \pendiente en producción
\pendiente{Agregar bibliografía de la asignatura.}

\subsection*{Referencias Básicas}

%* Ejemplo de referencia a un libro impreso
\hangindent=0.5in
Apellido, N. (Año). \textit{Título del libro}. Editorial.

%* Ejemplo de referencia a un libro en línea
\hangindent=0.5in
Apellido, N. (Año). \textit{Título del libro}. Editorial. \href{https://www.ejemplo.com}{https://www.ejemplo.com}

%* Ejemplo de referencia a un capítulo de un libro
\hangindent=0.5in
Apellido, A. y Apellido, B. (Año). Título del capítulo. En N. Apellido (Ed.), \textit{Título del libro} (pp. xx-xx). Editorial.

%* Ejemplo de página web con contenido estático
\hangindent=0.5in
Apellido, A., Apellido, B., y Apellido, C. (20 de mayo de 2020). \textit{Título del artículo de la página web}. Nombre del sitio web. \href{https://www.ejemplo.com}{https://www.ejemplo.com}

%* Ejemplo de página web con actualizaciones frecuentes
\hangindent=0.5in
Apellido, A., Apellido, B., y Apellido, C. (20 de mayo de 2020). \textit{Título del artículo de la página web}. Nombre del sitio web. Recuperado el día mes año de \href{https://www.ejemplo.com}{https://www.ejemplo.com}

%* Ejemplo de archivo con formato especial adentro de una página web
\hangindent=0.5in
Apellido, A. (03 de agosto de 2020). \textit{Título del archivo} [Archivo Excel]. Nombre del sitio web. \href{https://www.ejemplo.com}{https://www.ejemplo.com}

%* Ejemplo de vídeo de YouTube
\hangindent=0.5in
Nombre del autor. [Nombre de usuario en YouTube] (fecha). \textit{Título del video} [Video]. YouTube. \href{http://youtube.com/url-del-video}{http://youtube.com/url-del-video}

\subsection*{Referencias Complementarias}

%* Ejemplo de referencia a un libro impreso
\hangindent=0.5in
Apellido, N. (Año). \textit{Título del libro}. Editorial.

%* Ejemplo de referencia a un libro en línea
\hangindent=0.5in
Apellido, N. (Año). \textit{Título del libro}. Editorial. \href{https://www.ejemplo.com}{https://www.ejemplo.com}

%* Ejemplo de referencia a un capítulo de un libro
\hangindent=0.5in
Apellido, A. y Apellido, B. (Año). Título del capítulo. En N. Apellido (Ed.), \textit{Título del libro} (pp. xx-xx). Editorial.

%* Ejemplo de página web con contenido estático
\hangindent=0.5in
Apellido, A., Apellido, B., y Apellido, C. (20 de mayo de 2020). \textit{Título del artículo de la página web}. Nombre del sitio web. \href{https://www.ejemplo.com}{https://www.ejemplo.com}

%* Ejemplo de página web con actualizaciones frecuentes
\hangindent=0.5in
Apellido, A., Apellido, B., y Apellido, C. (20 de mayo de 2020). \textit{Título del artículo de la página web}. Nombre del sitio web. Recuperado el día mes año de \href{https://www.ejemplo.com}{https://www.ejemplo.com}

%* Ejemplo de archivo con formato especial adentro de una página web
\hangindent=0.5in
Apellido, A. (03 de agosto de 2020). \textit{Título del archivo} [Archivo Excel]. Nombre del sitio web. \href{https://www.ejemplo.com}{https://www.ejemplo.com}

%* Ejemplo de vídeo de YouTube
\hangindent=0.5in
Nombre del autor. [Nombre de usuario en YouTube] (fecha). \textit{Título del video} [Video]. YouTube. \href{https://www.youtube.com/url-del-video}{https://www.youtube.com/url-del-video}
\pagebreak