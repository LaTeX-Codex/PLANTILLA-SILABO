% Sección revisada: Estrategias Metodológicas
\section{Estrategias y Metodologías Didácticas}
Durante el desarrollo de la asignatura, se aplicará un diseño coherente que articule:
\begin{enumerate}[leftmargin=*, itemsep=0.6em]
    \item \textbf{Metodologías didácticas} (enfoque global).
    \item \textbf{Estrategias didácticas} (plan para unidades o bloques).
    \item \textbf{Técnicas y herramientas} (procedimientos y recursos puntuales).
\end{enumerate}

\subsection*{Metodologías didácticas}
Son los marcos teórico-prácticos que guían todo el proceso formativo. Para esta asignatura se adoptarán:
\begin{itemize}[leftmargin=*, itemsep=0.6em]
    \item \textbf{Aprendizaje Activo}: paradigma constructivista centrado en la participación y reflexión del estudiante.
    \item \textbf{Aprendizaje Basado en Proyectos (ABP)}: los estudiantes desarrollan proyectos reales o simulados para aplicar conceptos en contextos auténticos.
    \item \textbf{Aprendizaje Basado en Problemas (ABPr)}: se plantean problemas complejos y abiertos que los alumnos investigan y resuelven en grupos.
    \item \textbf{Gamificación}: incorporación de dinámicas de juego (puntos, niveles, recompensas) para motivar y comprometer a los estudiantes.
\end{itemize}

\subsection*{Estrategias didácticas}
Planificaciones adaptadas a cada unidad temática, donde se combinarán técnicas concretas para alcanzar objetivos específicos:
\begin{itemize}[leftmargin=*, itemsep=0.6em]
    \item \textbf{Aula Invertida}: contenidos teóricos en casa (videos, lecturas) y actividades prácticas en clase para resolver dudas y profundizar.
    \item \textbf{Estudio de Casos}: análisis de escenarios reales o hipotéticos, discusión en grupo y toma de decisiones fundamentadas.
    \item \textbf{Simulaciones y Juegos de Rol}: recreación de situaciones profesionales para practicar negociación, resolución de conflictos y toma de decisiones en un entorno controlado.
\end{itemize}

\subsection*{Técnicas y herramientas}
Son procedimientos y recursos específicos que implementan las estrategias:
\begin{itemize}[leftmargin=*, itemsep=0.6em]
    \item \textbf{Uso de Tecnologías Educativas}: plataformas virtuales, aplicaciones y recursos digitales que facilitan la comunicación, la interacción y el aprendizaje autónomo.
    \item \textbf{Retroalimentación Continua e Inmediata}: mecanismos para que los estudiantes reciban comentarios sobre su desempeño en tiempo real, favoreciendo la autoevaluación y el aprendizaje reflexivo.
\end{itemize}
\pagebreak