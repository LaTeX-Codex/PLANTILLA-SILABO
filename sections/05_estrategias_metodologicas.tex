\section{Estrategias Metodológicas}

Las actividades, métodos y técnicas para lograr un resultado de aprendizaje efectivo, se deben realizar usando las diferentes estrategias que se enumeran a continuación:

\textbf{Interculturalidad y género}, este curso abordará la interculturalidad tomando las cualidades siguientes:\

\begin{itemize}
    \item Promover actitudes y capacidades de aceptación, respeto e interacción en la diversidad cultural y género.
    \item Mediar conflictos a través de la escucha activa, empatía y ética para propiciar clima de convivencia y tranquilidad.
    \item Comprender al estudiantado brindándole confianza en el desarrollo de las actividades académicas que potencien su formación técnica, profesional y humanística.
    \item Garantizar las oportunidades con igualdad de condiciones al estudiantado para que demuestren sus capacidades, habilidades y destrezas acompañándolos en las experiencias didácticas.
    \item Promover estrategias de trabajo colaborativos y cooperativos que favorezcan las relaciones intrapersonales e interpersonales que contribuyan a eliminar posibles estereotipos.
    \item Promover el empoderamiento de hombres y mujeres vulnerables.
\end{itemize}

\textbf{Conferencias}, donde se pueden abordar todos los aspectos teóricos y se pueden aclarar al mismo tiempo aspectos iniciales y metodología a cumplir durante cada sesión de clases.

\textbf{Talleres}, donde se le facilite al estudiantado el camino a seguir para aplicar los conocimientos técnicos adquiridos durante la sesión teórica, donde se puede verificar si el estudiante ha asimilado cada contenido, donde el método de presentación es libre, dejando como opciones los reportes, videos y presentación en vivo.

\textbf{Tecnologías de la información y comunicaciones (TIC)}, basado en dispositivos tecnológicos e inteligentes con acceso a una amplia variedad de fuentes de información, a la vez del uso de software educativo para fortalecer la enseñanza aprendizaje con herramientas innovadoras, tales como la plataforma educativa MOODLE, Telegram como método de comunicación directa, asimismo, la plataforma de alojamiento de código fuente GitHub.

\textbf{Exposiciones}, que permitan al estudiante indagar sobre aspectos teóricos y técnicos referentes al proceso de diseño y escritura de páginas web, así como la utilización adecuada y buen manejo de las tecnologías web.

\textbf{Resolución de retos prácticos} apoyándose en la guía metodológica otorgada por el docente, donde tendrán que resolverla en un plazo de tiempo acorde a la dificultad del reto, para entregar resultado al docente, donde cada estudiante, durante la sesión de clases, analizará la guía y compartirá sus dudas.

\textbf{Retroalimentación} de cada clase anterior con el objetivo de establecer vínculos en serie entre cada una, donde este espacio se usará para enfocarse en elementos que dificultaron en la práctica.

\pagebreak