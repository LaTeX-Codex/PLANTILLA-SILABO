\section{Estrategias Metodológicas}


A continuación, se presenta una propuesta híbrida para la clase de Seguridad Aplicada a la Informática, en la que se combinan estrategias tradicionales actualizadas con enfoques innovadores. El objetivo es transformar el proceso de enseñanza en una experiencia inmersiva, colaborativa y motivadora, en la que se integren simulaciones, gamificación, actividades de metacognición y eventos tipo Capture The Flag (CTF).

La propuesta parte de un enfoque humanístico e inclusivo, en el que se valoran y respetan las diversidades culturales, de género y los distintos estilos de aprendizaje. Se busca crear un ambiente en el que los estudiantes se sientan acogidos y puedan expresar sus ideas libremente. Para ello, se promueve la comunicación asertiva y la resolución de conflictos a través de actividades como role-playing y debates, que ayudan a fortalecer la empatía y el compromiso ético. Además, se fomenta el trabajo en equipo mediante la conformación de grupos heterogéneos que permitan compartir perspectivas y desarrollar habilidades interpersonales y técnicas de manera simultánea.

Las sesiones teóricas se llevan a cabo de forma interactiva, combinando el modelo de aula invertida con técnicas de microaprendizaje. Antes de cada clase, se facilita a los estudiantes el acceso a contenidos multimedia –como videos, podcasts y lecturas interactivas– que les permitan familiarizarse con los temas. Este enfoque permite que el tiempo en el aula se dedique a la discusión, a la aplicación práctica y a la resolución de problemas reales, utilizando herramientas de respuesta en tiempo real que ayudan a verificar la comprensión y a fomentar la participación activa.

El componente práctico del curso se refuerza mediante talleres y laboratorios virtuales, en los que se utilizan simuladores para recrear escenarios reales de ataques y defensas en sistemas informáticos. En estos espacios, los estudiantes pueden poner a prueba sus conocimientos en un entorno seguro y controlado, aplicando técnicas de análisis, detección y mitigación de vulnerabilidades. Las simulaciones se integran con estudios de caso y talleres que permiten establecer conexiones directas entre la teoría y su aplicación en situaciones reales.

Un elemento central e innovador en esta propuesta es la incorporación de la gamificación y la realización de eventos tipo Capture The Flag (CTF). Se diseñan actividades lúdicas y competitivas que transforman el aprendizaje en una experiencia motivadora. El sistema de gamificación se basa en la asignación de puntos, niveles y recompensas simbólicas, reconociendo el esfuerzo y el logro de competencias técnicas a medida que los estudiantes resuelven desafíos en tiempo real. Los eventos CTF, organizados de forma individual o en equipos, desafían a los participantes a identificar y explotar vulnerabilidades en entornos simulados, fomentando el pensamiento crítico, la creatividad y el trabajo colaborativo.

La integración de tecnologías es otro pilar fundamental de esta propuesta. Se emplean plataformas educativas como MOODLE para gestionar contenidos y evaluaciones en línea, y repositorios colaborativos como GitHub para facilitar la documentación y revisión de proyectos. Además, se utilizan aplicaciones de mensajería y, en algunos casos, chatbots o aplicaciones móviles que proporcionan asistencia inmediata durante las actividades prácticas. Estas herramientas permiten también el seguimiento del desempeño de los estudiantes, facilitando una retroalimentación automatizada que ayuda a ajustar las estrategias pedagógicas en función de las necesidades de cada grupo.

Por último, se pone especial énfasis en la evaluación formativa y en el desarrollo de habilidades metacognitivas. A lo largo del curso se realizan evaluaciones continuas mediante quizzes, autoevaluaciones y evaluaciones entre pares, que ofrecen retroalimentación inmediata y permiten identificar áreas de mejora. Los estudiantes mantienen portafolios digitales en los que registran su progreso, reflexionan sobre sus aprendizajes y planifican estrategias para superar desafíos. Al final de cada unidad o evento significativo, como los hackathons o CTF, se realizan sesiones de reflexión en las que se analizan tanto los aciertos como las áreas que requieren mayor atención, consolidando así el ciclo de mejora continua y promoviendo la autorreflexión.

En resumen, esta propuesta híbrida combina la solidez de métodos tradicionales –actualizados con simulaciones y el uso intensivo de tecnologías– con elementos innovadores que incluyen gamificación, actividades metacognitivas y eventos CTF. El resultado es un ambiente de aprendizaje integral y dinámico, en el que los estudiantes no solo adquieren competencias técnicas en seguridad informática, sino que también desarrollan habilidades blandas, éticas y colaborativas, preparándolos de manera óptima para los desafíos del entorno profesional actual.