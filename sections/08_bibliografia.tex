\section{Bibliografía}
%* Aquí debe poner la bibliografía de la asignatura
%! Use el formato APA para las referencias

%! Eliminar este \pendiente en producción
\pendiente{Agregar bibliografía de la asignatura.}

\subsection*{Referencias Básicas}

%! Ejemplos de libros impresos
\hangindent=1in
Merino, C., \& Cañizares, R. (2011). \textit{Implementación de un Sistema de Gestión de Seguridad de la Información según ISO 27001}. Fundación Confemetal.

\hangindent=1in
Peltier, J. (2020). \textit{Information security policies, procedures, and standards: Guidelines for effective information security management} (3ª ed.). Auerbach Publications.

%! Ejemplos de artículos de revistas
\hangindent=1in
Arévalo, F. M., \& Cedillo, L. P. (2017). \textit{Metodología Ágil para la gestión de Riesgos Informáticos}. Killkana Técnica, 1(2), 45-67.

\hangindent=1in
ISO/IEC. (2013). \textit{ISO/IEC 27005:2013 Information technology - Security techniques - Information security risk management}. ISO Bulletin, 4(2), 12-18. \href{https://doi.org/10.1016/j.isac.2013.05.002}{https://doi.org/10.1016/j.isac.2 013.05.002}

%! Ejemplos de capítulos de libros
\hangindent=1in
Bautista Torres, L. A. (2018). Seguridad de la información en entornos empresariales. En J. González \& M. López (Eds.), \textit{Gestión de riesgo en la seguridad informática} (pp. 78-105). Editorial Tecnológica.

\hangindent=1in
Imbaquinto, Esparza, D. E., Pusda, Chulde, M. R., \& Jácome León, J. G. (2016). Fundamento de Auditoría Basados en Riesgos. En \textit{Auditoría informática en entornos digitales} (pp. 112-134). UTN.

\subsection*{Referencias Complementarias}
\pagebreak