\documentclass[letterpaper, 12pt, titlepage]{article} % Tipo de documento, configurado con: tamaño carta, tamaño de fuente 12pt, inclusión de página de título

% ================================================================
% PAQUETES BÁSICOS Y CONFIGURACIÓN ESENCIAL
% ================================================================
\usepackage[utf8]{inputenc}         % Codificación de caracteres UTF-8
\usepackage[spanish]{babel}         % Localización en español (guiones, títulos)
\usepackage[margin=1in]{geometry}   % Configuración de márgenes de página

% ================================================================
% TIPOGRAFÍA Y FORMATO DE TEXTO
% ================================================================
%\usepackage{fontspec}              % Manejo de fuentes con XeLaTeX
%\setmainfont{Myriad Pro} % Fuente principal (Myriad Pro)
\usepackage{helvet}                         % Fuente Sans-Serif (Helvetica-like)
\renewcommand{\familydefault}{\sfdefault}   % Establece Helvet como fuente principal
\usepackage{setspace}                       % Control de interlineado (1.5, doble espacio)
\usepackage{parskip}                        % Espaciado entre párrafos (en lugar de sangría)
\usepackage{ragged2e}                       % Mejor manejo de justificación de texto

\renewcommand{\thesection}{\Roman{section}} % Numeración romana para secciones
\renewcommand{\thesubsection}{\arabic{section}.\arabic{subsection}} % Subsecciones: 1.1, 1.2...
\renewcommand{\thesubsubsection}{\arabic{section}.\arabic{subsection}.\arabic{subsubsection}} % Subsubsecciones: 1.1.1, 1.1.2...

% ================================================================
% ELEMENTOS GRÁFICOS Y DISEÑO
% ================================================================
\usepackage{graphicx}               % Manejo de imágenes (inserción y escalado)
\usepackage{xcolor}                 % Sistema de colores personalizados
\usepackage{pdflscape}              % Páginas en orientación horizontal

% ================================================================
% TABLAS PROFESIONALES
% ================================================================
\usepackage{booktabs}              % Reglas tipográficas para tablas (formato APA)
\usepackage{tabularx}              % Tablas con ancho ajustable
\usepackage{xltabular}             % Tablas largas que ocupan todo el ancho del texto
\usepackage{makecell}              % Celdas personalizadas en tablas
% ================================================================
% LISTAS Y ESTRUCTURAS
% ================================================================
\usepackage{enumitem}                   % Personalización avanzada de listas
\setlist{nosep, leftmargin=*, wide=0pt} % Listas ultra compactas

% ================================================================
% HIPERVÍNCULOS Y METADATOS PDF
% ================================================================
\usepackage[hyphens]{url}
\usepackage{hyperref}             % Hipervínculos y referencias internas
\hypersetup{
    breaklinks=true,                % Configuración de hipervínculos
    colorlinks=true,                % Habilita los enlaces coloreados
    linkcolor=blue,                 % Color de los enlaces internos
    urlcolor=blue,                  % Color de los enlaces externos
    citecolor=blue,                 % Color de las citas
    pdfborder={0 0 0}               % Sin bordes en los enlaces
}

% ================================================================
% TO DO: TAREAS PENDIENTES Y ANOTACIONES
% ================================================================
\setlength{\marginparwidth}{2cm} % Soluciona el problema de todonotes
\usepackage{todonotes} % Paquete principal
\setuptodonotes{
    size=\tiny, % Tamaño de las notas
}
% Comandos personalizados (opcional)
\newcommand{\pendiente}[1]{\todo[color=red!30, inline]{#1}}
\newcommand{\revisar}[1]{\todo[color=yellow!30, inline]{#1}}
\newcommand{\actualizar}[1]{\todo[color=green!30, inline]{#1}}

% ================================================================
% CONFIGURACIONES FINALES
% ================================================================
\newcommand{\universidad}{Universidad de las Regiones Autónomas de la Costa Caribe Nicaragüense} % Nombre de la universidad
\newcommand{\carrera}{Nombre de la Carrera} % Nombre de la carrera
\newcommand{\asignatura}{Nombre de la Asignatura} % Nombre de la asignatura
\newcommand*{\encabezadodetabla}[1]{\multicolumn{1}{>{\centering\arraybackslash}X}{\bfseries #1}} % Encabezado centrado y en negritas en tablas

% ================================================================
% PAQUETE LOREM IPSUM PARA TEXTO DE EJEMPLO, BORRAR EN PRODUCCIÓN
% ================================================================
\usepackage{lipsum} % Generador de texto de ejemplo

% ================================================================
% INICIO DEL DOCUMENTO (contenido principal)
% ================================================================
\begin{document}

% Se incluyen las secciones del syllabus que se encuentran en el directorio sections
\begin{titlepage}
    \noindent
    \begin{minipage}{0.3\textwidth}
        \begin{center}
            \includegraphics[width=0.6\textwidth]{images/uraccan_logo.png}
        \end{center}
    \end{minipage}%
    \begin{minipage}{0.6\textwidth}
        \begin{center}
            \Large{\textbf{Universidad de las Regiones Autónomas\\de la Costa Caribe Nicaragüense}}\\
            \Large{\textbf{URACCAN-CUR Las Minas}}\\
        \end{center}
    \end{minipage}
\end{titlepage}
\section{Datos Generales}
\begin{doublespace}
    \begin{tabularx}{\linewidth}{@{}>{\bfseries}l X@{}}
        Carrera:                 & \carrera                                                                \\
        Plan de estudio:         & 20XX                                                                    \\
        Asignatura:              & \asignatura                                                             \\
        Código de la asignatura: & XXX-XXX                                                                 \\
        Requisito:               & Nombre de la asignatura requisito                                       \\
        Régimen:                 & Régimen académico                                                       \\
        Semestre/Cuatrimestre:   & X Semestre/X Cuatrimestre                                               \\
        Modalidad                & Presencial                                                              \\
        Frecuencia semanal:      & X horas semanales                                                       \\
        Turno:                   & Diurno/Nocturno                                                         \\
        Horas:                   & X horas (X horas de atención directa, X horas de trabajo independiente) \\
        Año Académico:           & X Año                                                                   \\
        Autor:                   & Nombre del docente                                                      \\
    \end{tabularx}
\end{doublespace}
\pagebreak
% Para poder usar footnotes en el título de la sección:
% \SECTION[Título interno de la sección]{Título visible de la sección}
\section[Descriptor de la asignatura]{Descriptor de la asignatura\footnote{Texto íntegro del descriptor, tomado del programa de la asignatura.}}

% Aquí debe poner el texto del descriptor de la asignatura, se encuentra en el programa de la asignatura.

% Texto Lorem Ipsum para el descriptor de la asignatura. Eliminar en producción.
\lipsum[1-4] % Genera cuatro párrafos de texto de ejemplo
\pagebreak
% Para poder usar \footnotes en el título de la sección:
% \SECTION[Título interno de la sección]{Título visible de la sección}
\section[Objetivos Generales o Resultados de Aprendizaje]{Objetivos Generales o Resultados de Aprendizaje\footnote{Texto íntegro de los objetivos o resultados de aprendizaje, tomado del programa de la asignatura.}}
%Eliminar \PENDIENTE en producción.
\pendiente{Según el enfoque de la asignatura, se pueden definir objetivos generales o resultados de aprendizaje. Pero no los dos a la vez. Revise el programa de la asignatura.}

%Eliminar \PENDIENTE en producción.
\pendiente{La siguiente tabla es en caso de que sea una asignatura por objetivos. Si es por competencias, elimine esta tabla y use la siguiente.}

\small
\begin{xltabular}{\linewidth}{@{}X X X@{}}
    \toprule
    \thead{\small{\textbf{Conceptuales}}} & \thead{\small{\textbf{Procedimentales}}} & \thead{\small{\textbf{Actitudinales}}} \\
    \midrule
    \begin{itemize}[labelsep=2pt, itemsep=0.6em]
        \item \lipsum[1][1-3] % Texto de ejemplo, eliminar en producción
        \item \lipsum[2][1-3] % Texto de ejemplo, eliminar en producción
        \item \lipsum[3][1-3] % Texto de ejemplo, eliminar en producción
        \item \lipsum[4][1-3] % Texto de ejemplo, eliminar en producción
    \end{itemize} &
    \begin{itemize}[labelsep=2pt, itemsep=0.6em]
        \item \lipsum[1][1-3] % Texto de ejemplo, eliminar en producción
        \item \lipsum[2][1-3] % Texto de ejemplo, eliminar en producción
        \item \lipsum[3][1-3] % Texto de ejemplo, eliminar en producción
        \item \lipsum[4][1-3] % Texto de ejemplo, eliminar en producción
    \end{itemize} &
    \begin{itemize}[labelsep=2pt, itemsep=0.6em]
        \item \lipsum[1][1-3] % Texto de ejemplo, eliminar en producción
        \item \lipsum[2][1-3] % Texto de ejemplo, eliminar en producción
        \item \lipsum[3][1-3] % Texto de ejemplo, eliminar en producción
        \item \lipsum[4][1-3] % Texto de ejemplo, eliminar en producción
    \end{itemize} \\
    \bottomrule
\end{xltabular}

%Eliminar \PENDIENTE en producción.
\pendiente{La siguiente tabla es en caso de que sea una asignatura por competencias. Si es por objetivos, elimine esta tabla y use la primera.}
\pendiente{Investigue el formato de la tabla de competencias en el programa de la asignatura.}
\pagebreak
\begin{landscape}
    \section{Contenidos}

    \begin{doublespace}
        \begin{tabularx}{\linewidth}{@{}c@{}l@{}c@{}c@{}c@{}c@{}}
            \toprule
            \thead{No.}                               & \thead{Unidad}          & \thead{Horas Teóricas} & \thead{Horas Prácticas} & \thead{Horas de Trabajo Independiente} & \thead{Total de Horas} \\
            \midrule
            I                                         & Nombre de la unidad I   & 4                      & 4                       & 16                                     & 24                     \\
            II                                        & Nombre de la unidad II  & 4                      & 4                       & 16                                     & 24                     \\
            III                                       & Nombre de la unidad III & 4                      & 4                       & 16                                     & 24                     \\
            IV                                        & Nombre de la unidad IV  & 4                      & 4                       & 16                                     & 24                     \\
            V                                         & Nombre de la unidad V   & 4                      & 4                       & 16                                     & 24                     \\
            VI                                        & Nombre de la unidad VI  & 4                      & 4                       & 16                                     & 24                     \\
            \midrule
            \multicolumn{2}{c}{\textbf{Evaluaciones}} & 4                       & 4                      & 16                      & 24                                                              \\
            \midrule
            \multicolumn{2}{c}{\textbf{Total}}        & \textbf{32}             & \textbf{32}            & \textbf{128}            & \textbf{192}                                                    \\
            \bottomrule
        \end{tabularx}
    \end{doublespace}
\end{landscape}
\pagebreak
\section{Estrategias Metodológicas}


A continuación, se presenta una propuesta híbrida para la clase de Seguridad Aplicada a la Informática, en la que se combinan estrategias tradicionales actualizadas con enfoques innovadores. El objetivo es transformar el proceso de enseñanza en una experiencia inmersiva, colaborativa y motivadora, en la que se integren simulaciones, gamificación, actividades de metacognición y eventos tipo Capture The Flag (CTF).

La propuesta parte de un enfoque humanístico e inclusivo, en el que se valoran y respetan las diversidades culturales, de género y los distintos estilos de aprendizaje. Se busca crear un ambiente en el que los estudiantes se sientan acogidos y puedan expresar sus ideas libremente. Para ello, se promueve la comunicación asertiva y la resolución de conflictos a través de actividades como role-playing y debates, que ayudan a fortalecer la empatía y el compromiso ético. Además, se fomenta el trabajo en equipo mediante la conformación de grupos heterogéneos que permitan compartir perspectivas y desarrollar habilidades interpersonales y técnicas de manera simultánea.

Las sesiones teóricas se llevan a cabo de forma interactiva, combinando el modelo de aula invertida con técnicas de microaprendizaje. Antes de cada clase, se facilita a los estudiantes el acceso a contenidos multimedia –como videos, podcasts y lecturas interactivas– que les permitan familiarizarse con los temas. Este enfoque permite que el tiempo en el aula se dedique a la discusión, a la aplicación práctica y a la resolución de problemas reales, utilizando herramientas de respuesta en tiempo real que ayudan a verificar la comprensión y a fomentar la participación activa.

El componente práctico del curso se refuerza mediante talleres y laboratorios virtuales, en los que se utilizan simuladores para recrear escenarios reales de ataques y defensas en sistemas informáticos. En estos espacios, los estudiantes pueden poner a prueba sus conocimientos en un entorno seguro y controlado, aplicando técnicas de análisis, detección y mitigación de vulnerabilidades. Las simulaciones se integran con estudios de caso y talleres que permiten establecer conexiones directas entre la teoría y su aplicación en situaciones reales.

Un elemento central e innovador en esta propuesta es la incorporación de la gamificación y la realización de eventos tipo Capture The Flag (CTF). Se diseñan actividades lúdicas y competitivas que transforman el aprendizaje en una experiencia motivadora. El sistema de gamificación se basa en la asignación de puntos, niveles y recompensas simbólicas, reconociendo el esfuerzo y el logro de competencias técnicas a medida que los estudiantes resuelven desafíos en tiempo real. Los eventos CTF, organizados de forma individual o en equipos, desafían a los participantes a identificar y explotar vulnerabilidades en entornos simulados, fomentando el pensamiento crítico, la creatividad y el trabajo colaborativo.

La integración de tecnologías es otro pilar fundamental de esta propuesta. Se emplean plataformas educativas como MOODLE para gestionar contenidos y evaluaciones en línea, y repositorios colaborativos como GitHub para facilitar la documentación y revisión de proyectos. Además, se utilizan aplicaciones de mensajería y, en algunos casos, chatbots o aplicaciones móviles que proporcionan asistencia inmediata durante las actividades prácticas. Estas herramientas permiten también el seguimiento del desempeño de los estudiantes, facilitando una retroalimentación automatizada que ayuda a ajustar las estrategias pedagógicas en función de las necesidades de cada grupo.

Por último, se pone especial énfasis en la evaluación formativa y en el desarrollo de habilidades metacognitivas. A lo largo del curso se realizan evaluaciones continuas mediante quizzes, autoevaluaciones y evaluaciones entre pares, que ofrecen retroalimentación inmediata y permiten identificar áreas de mejora. Los estudiantes mantienen portafolios digitales en los que registran su progreso, reflexionan sobre sus aprendizajes y planifican estrategias para superar desafíos. Al final de cada unidad o evento significativo, como los hackathons o CTF, se realizan sesiones de reflexión en las que se analizan tanto los aciertos como las áreas que requieren mayor atención, consolidando así el ciclo de mejora continua y promoviendo la autorreflexión.

En resumen, esta propuesta híbrida combina la solidez de métodos tradicionales –actualizados con simulaciones y el uso intensivo de tecnologías– con elementos innovadores que incluyen gamificación, actividades metacognitivas y eventos CTF. El resultado es un ambiente de aprendizaje integral y dinámico, en el que los estudiantes no solo adquieren competencias técnicas en seguridad informática, sino que también desarrollan habilidades blandas, éticas y colaborativas, preparándolos de manera óptima para los desafíos del entorno profesional actual.
\section{Materiales y Recursos Didácticos}

\subsection*{Materiales físicos y espacios}

\begin{xltabular}{\linewidth}{@{}>{\bfseries}X X@{}}
    \toprule
    Pizarra acrílica & Superficie para escritura y explicaciones en tiempo real. \\
    \midrule
    Pantalla o proyector & Dispositivo de visualización de contenido digital/audiovisual. \\
    \midrule
    Laboratorio de computación & Espacio equipado con equipos tecnológicos y acceso a internet para ejercicios prácticos. \\
    \midrule
    Computadoras & Dispositivos para ejecutar software, acceder a recursos digitales, realización de prácticas y elaboración de informes. \\
    \bottomrule
\end{xltabular}

\subsection*{Documentación}

\begin{xltabular}{\linewidth}{@{}>{\bfseries}X X@{}}
    \toprule
    Sílabo del curso & Documento que detalla los objetivos, el contenido y el cronograma del curso. \\
    \midrule
    Planes de clase & Organización detallada de actividades y objetivos por sesión. \\
    \midrule
    Presentaciones de diapositivas & Recursos visuales para explicar conceptos clave y facilitar el aprendizaje. \\
    \midrule
    Material bibliográfico & Libros, artículos y recursos digitales que fundamentan el contenido del curso. \\
    \midrule
    Guías de laboratorio & Documento que contiene los procedimientos para las prácticas técnicas y simulaciones. \\
    \midrule
    Ejercicios prácticos guiados & Actividades estructuradas con instrucciones paso a paso para aplicar los conocimientos adquiridos. \\
    \midrule
    Bibliografía de apoyo & Libros y textos complementarios para profundizar en los temas tratados en el curso. \\
    \bottomrule
\end{xltabular}

\subsection*{Herramientas digitales y software}

\begin{xltabular}{\linewidth}{@{}>{\bfseries}X X@{}}
    \toprule
    Plataforma educativa MOODLE & Entorno virtual para la gestión de cursos, materiales y actividades. \\
    \midrule
    Sistema operativo (Windows/Linux) & Base para el funcionamiento de las computadoras y ejecución de software. \\
    \midrule
    Microsoft Word, Google Docs, LibreOffice Writer o LaTeX & Herramientas para la creación y edición de documentos textuales. \\
    \midrule
    Microsoft PowerPoint, Google Slides o LibreOffice Impress & Software para diseñar presentaciones multimedia. \\
    \midrule
    Telegram & Aplicación de mensajería para comunicación informal y rápida con los estudiantes. \\
    \midrule
    Microsoft Teams & Plataforma para videoclases, reuniones y colaboración institucional. \\
    \bottomrule
\end{xltabular}
\pagebreak
\section{Metodología de Evaluación}

La aprobación satisfactoria del curso depende de una nota mayor o igual a 60 puntos que resulta del promedio de dos evaluaciones parciales a lo largo del curso.

La primera prueba parcial está estructurada en 60\% de evaluación sumativa y 40\% de evaluación formativa. La evaluación formativa estará dividida en dos partes: un examen teórico a realizarse en el aula virtual; y la defensa de una práctica de laboratorio basada en el contenido práctico visto en las prácticas de laboratorio del parcial.

La segunda prueba parcial también se estructura del 60\% de evaluación sumativa y 40\% de evaluación formativa enfocado en un proyecto final donde aplique todo lo aprendido.

Las evaluaciones sumativas se componen de prácticas de laboratorio y actividades de lectoescritura que resultan en exposiciones o informes de investigación de cada tema propuesto.

Las evaluaciones formativas se enfocarán en guiar al estudiante a trabajar en un proyecto de magnitud o dificultad considerable, aplicando de forma integral los conocimientos adquiridos en el transcurso de la asignatura a un caso simulado, dando una solución satisfactoria al reto presentado.

Es recomendable que las evaluaciones prácticas se valore el trabajo individual, para garantizar el aprendizaje total en cada uno.
\pagebreak
\section{Bibliografía}
%* Aquí debe poner la bibliografía de la asignatura
%! Use el formato APA para las referencias

%! Eliminar este \pendiente en producción
\pendiente{Agregar bibliografía de la asignatura.}

\subsection*{Referencias Básicas}

%! Ejemplos de libros impresos
\hangindent=1in
Merino, C., \& Cañizares, R. (2011). \textit{Implementación de un Sistema de Gestión de Seguridad de la Información según ISO 27001}. Fundación Confemetal.

\hangindent=1in
Peltier, J. (2020). \textit{Information security policies, procedures, and standards: Guidelines for effective information security management} (3ª ed.). Auerbach Publications.

%! Ejemplos de artículos de revistas
\hangindent=1in
Arévalo, F. M., \& Cedillo, L. P. (2017). \textit{Metodología Ágil para la gestión de Riesgos Informáticos}. Killkana Técnica, 1(2), 45-67.

\hangindent=1in
ISO/IEC. (2013). \textit{ISO/IEC 27005:2013 Information technology - Security techniques - Information security risk management}. ISO Bulletin, 4(2), 12-18. \href{https://doi.org/10.1016/j.isac.2013.05.002}{https://doi.org/10.1016/j.isac.2 013.05.002}

%! Ejemplos de capítulos de libros
\hangindent=1in
Bautista Torres, L. A. (2018). Seguridad de la información en entornos empresariales. En J. González \& M. López (Eds.), \textit{Gestión de riesgo en la seguridad informática} (pp. 78-105). Editorial Tecnológica.

\hangindent=1in
Imbaquinto, Esparza, D. E., Pusda, Chulde, M. R., \& Jácome León, J. G. (2016). Fundamento de Auditoría Basados en Riesgos. En \textit{Auditoría informática en entornos digitales} (pp. 112-134). UTN.

\subsection*{Referencias Complementarias}
\pagebreak
\begin{landscape}
    \section{Plan Calendario}

    \begin{adjustbox}{width=1.3847\textwidth}
        \begin{tabular}{ | p{3cm} | p{5cm} | p{5cm} | p{3cm} | p{3cm} | p{3cm} | p{5cm} | }
            \hline
            \textbf{Fecha}            & \textbf{Objetivos o Resultados de Aprendizaje} & \textbf{Unidad y Contenidos} & \textbf{Estrategias Metodológicas} & \textbf{Trabajo Independiente} & \textbf{Mecanismos de Evaluación} & \textbf{Bibliografía} \\
            \hline
            % Inicio Encuentro 1
            % Fecha
            03 de marzo 2025          &
            % Objetivos o Resultados de Aprendizaje
            \begin{minipage}
                [t]{5cm}
                \begin{itemize}[leftmargin=10pt]
                    \item Reconoce la importancia de la ética en la vida profesional.
                    \item Identifica los principios éticos que rigen la profesión.
                \end{itemize}
                \vspace{0.2cm}
            \end{minipage} &
            \begin{minipage}
                [t]{5cm}
                \begin{itemize}[leftmargin=10pt]
                    \item Unidad 1: Ética y Profesión.
                    \item Contenidos: Concepto de ética, ética profesional, principios éticos.
                \end{itemize}
                \vspace{0.2cm}
            \end{minipage} &
            \begin{minipage}
                [t]{3cm}
                \begin{itemize}[leftmargin=10pt]
                    \item Exposición.
                    \item Debate.
                    \item Análisis de casos.
                \end{itemize}
            \end{minipage} &
            \begin{minipage}
                [t]{3cm}
                \begin{itemize}[leftmargin=10pt]
                    \item Lectura de textos.
                    \item Investigación.
                \end{itemize}
            \end{minipage} &
            \begin{minipage}
                [t]{3cm}
                \begin{itemize}[leftmargin=10pt]
                    \item Participación en clase.
                    \item Exposición.
                    \item Análisis de casos.
                \end{itemize}
            \end{minipage} &
            \begin{minipage}
                [t]{5cm}
                \begin{itemize}[leftmargin=10pt]
                    \item \textit{Ética para Amador} - Fernando Savater.
                    \item \textit{Ética Profesional} - José A. Silié Ruiz.
                \end{itemize}
            \end{minipage}                                                                                                                                                                                                        \\
            \hline
            % Fin Encuentro 1
        \end{tabular}
    \end{adjustbox}
\end{landscape}

\end{document}
% ================================================================
% FIN DEL DOCUMENTO
% ================================================================